\documentclass{article}
\usepackage[utf8]{inputenc}

\usepackage[T2A]{fontenc}
\usepackage[utf8]{inputenc}
\usepackage[russian]{babel}

\usepackage{multienum}
\usepackage{geometry}
\usepackage{hyperref}

\geometry{
 left=1cm,right=1cm,
 top=2cm,bottom=2cm
}

\title{История}
\author{Лисид Лаконский}
\date{May 2023}

\newtheorem{definition}{Определение}

\begin{document}
\raggedright

\maketitle
\tableofcontents
\pagebreak

\section{Практическое занятие по истории №18, «Перестройка в СССР: от попыток совершенствования системы к смене модели общественного развития»}

\subsection{Состояние советской экономики в середине 80-х годов}

В середине 1980-х годов советская экономика столкнулась с серьезными проблемами. Хотя советская экономика в прошлом периоде переживала периоды роста и индустриализации, к середине 1980-х годов она столкнулась с различными структурными проблемами и неэффективностью.

\hfill

Одной из основных проблем была централизованная плановая система, которая стала все более бюрократической и неуклюжей. Государство контролировало практически все аспекты экономики, включая производственные показатели, цены и распределение ресурсов. Эта система часто приводила к неэффективности, неправильному управлению и отсутствию реакции на рыночные требования.

\hfill

Кроме того, советская экономика сильно зависела от производства природных ресурсов, особенно нефти и газа. Однако к середине 1980-х годов стали появляться признаки застоя в нефтяной отрасли. Добыча нефти стала более сложной и дорогостоящей, а производительность снижалась. Этот спад в добыче нефти оказал значительное влияние на общую экономику, поскольку нефтяные доходы играли важную роль в финансировании других отраслей и поддержке советского государства.

\hfill

Кроме того, существовали серьезные проблемы с производительностью и инновациями в советской экономике. Централизованная плановая система не поощряла индивидуальную инициативу и предпринимательство, что приводило к отсутствию стимулов для работников и предприятий улучшать эффективность или разрабатывать новые технологии. В результате рост производительности был медленным, а технологические достижения были ограничены по сравнению с другими индустриализованными странами.

\hfill

Состояние советской экономики в середине 1980-х годов также характеризовалось растущим дефицитом бюджета, увеличением внешнего долга и снижением качества товаров и услуг, доступных населению. Потребительские товары часто были в дефиците, а на строительство жилья и другие основные потребности требовались долгие ожидания.

\hfill

В целом, советская экономика в середине 1980-х годов столкнулась с множеством проблем, которые в конечном итоге способствовали ее краху. Неэффективность централизованной плановой системы, спад в добыче нефти, низкий рост производительности и нарастающие экономические проблемы сыграли роль в последующей экономической и политической трансформации России и распаде Советского Союза.

\pagebreak
\subsection{Стратегия «ускорения социально-экономического развития» и ее провал}

Ускорение — лозунг и политический курс Генерального секретаря ЦК КПСС Михаила Горбачёва, провозглашённый 23 апреля 1985 года, на апрельском пленуме ЦК КПСС, одно из ключевых направлений реформаторского курса («гласность — перестройка — ускорение»), проводившегося в Советском Союзе, в 1985—1991 годах.

\hfill

Как отмечал Николай Рыжков, разработка экономической программы «ускорение» была начата уже в 1983 году, как и сам термин «ускорение», родился в связи с рассмотрением в ЦК предложений Госплана СССР о перспективах развития экономики СССР до 2000 года.

\hfill

Термином «ускорение» обычно принято называть ранний этап горбачёвских реформ (1985—1987), когда принимаемые меры носили сугубо административный характер.

\hfill

Целью проводимого курса было ускорение социального и экономического развития Советского Союза. По своей сути термин «ускорение» означал признание отставания СССР в развитии[1] от ведущих промышленных государств и стран мира и являлся новой версией старого лозунга «догнать и перегнать».

Сам термин впервые был использован Юрием Владимировичем Андроповым ещё 22 ноября 1982 года на пленуме ЦК КПСС: «Намечено ускорить темпы развития экономики, увеличить абсолютные размеры прироста национального дохода… Напряжённые задания должны быть выполнены при сравнительно меньшем увеличении материальных затрат и трудовых ресурсов.»

Для всего этого требовалась модернизация экономики и политической системы: «Нужны революционные сдвиги — переход к принципиально новым технологическим системам, к технике последних поколений, дающих наивысшую эффективность» — заявил Горбачёв.

\hfill

Это первый лозунг из пакета, в который также входили «перестройка», «гласность» и «демократизация». 

\subsubsection{Реализация и результаты}

Развитие советского машиностроения стало одним из приоритетов ускорения, его сверхзадачей. На XXVII съезде КПСС в начале 1986 г. задачей № 1 было признано ускоренное развитие машиностроения, в котором усматривалась основа быстрого перевооружения всего народного хозяйства. Программа ускорения намечала опережающее (в 1,7 раз) развитие машиностроения по отношению ко всей промышленности и достижение им мирового уровня уже в конце 1990-х гг. На развитие машиностроения партия выделила 200 млрд рублей, в два раза больше, чем за предыдущие десять лет.

\hfill

Однако капиталовложения в тяжёлую промышленность, импортные закупки для неё не давали положительного эффекта и на товарном и продовольственном рынке не сказывались. Более того, этот рынок стал жертвой «ускорения» развития машиностроения, так как импортные поставки для последнего вели к сокращению закупок продовольствия и товаров народного потребления, что привело к росту бюджетного дефицита, скрытой инфляции и товарному голоду[2].

\hfill

В 1987 году был признан фактический провал политики ускорения, и на смену ей пришла экономическая реформа, предполагавшая существенное расширение самостоятельности государственных предприятий и развитие частного сектора. В дальнейшем была разработана программа «500 дней».

\pagebreak
\subsection{Политическая борьба вокруг альтернатив экономического развития в 1987–1991 годы}

Экономические реформы в ходе перестройки можно разделить на три условных периода. Первый - «ускорение» (апрель 1985-1986г). Второй период экономических реформ - «перестройка» (1987г.- середина 1990г.) Третий период - переход к регулируемому рынку (июнь 1990г. - декабрь 1991г.)

\hfill

Итогом экономического реформирования первых двух периодов стало дальнейшее ухудшение экономического и финансового положения страны. Падение национального дохода в 1990г. по сравнению с 1989г. составило 9\%. Власть для поддержания жизненного уровня населения была вынуждена прибегнуть к массированным внешним займам. Именно в это время образовалось большая часть внешнего долга СССР- ответственность, за который в последствие легла на Россию.

\hfill

Экономическая ситуация продолжала ухудшаться, кризисные явления все более усиливались. Продолжалось падение общественного производства, снижение его эффективности, усиливалась денежно-финансовая несбалансированность, возрастала эмиссия денег, шел заметный рост цен на товары народного потребления, еще более обострялась ситуация на потребительском рынке, оставалось отрицательным сальдо внешней торговли и т. д. Сказывалось нарушение хозяйственных связей, ухудшение трудовой и договорной дисциплины и т. п.

\hfill

Политическая номенклатура понимала что без каких либо кардинальных мер экономику не спасти. Страх потерять власть заставил ее переметнуться на позиции рыночной экономики. Критическое положение в этой сфере и определенная растерянность горбачевского руководства, явно не знавшего, что делать, привели к развертыванию в 1989-1990 гг. экономической дискуссии. Было разработано и представлено десяток крупных экономических программ.

\hfill

Правительство предложило свою программу перехода к "планово-рыночной экономике" - "О мерах по оздоровлению экономики, этапах экономической реформы и принципиальных подходах к разработке XIII пятилетнего плана". Она определяла преодоление экономических трудностей (в первую очередь - бюджетного дефицита и разбалансированности потребительского рынка) и переход к нормальному функционированию экономики, на что требовалось шесть лет. На первом этапе (1990-1992 годы) должен быть, по мнению правительства, осуществлен комплекс чрезвычайных мер. В это время предполагалось использовать как директивные методы управления, так и экономические рычаги, роль которых должна была постепенно возрастать. На втором этапе (1993-1995 годы) ведущее место отводилось экономичным методам руководства. Более активно намечалось развивать рыночные отношения.

\hfill

Помимо правительственной был ряд альтернативных программ, в том числе программа межрегиональной депутатской группы, которая предлагала более радикальные меры и ускоренный переход к рыночной экономике. Для нормализации потребительского рынка и обеспечения социальной защищенности трудящихся с низкими доходами предлагалось создать два сектора на рынке: 1) ввести карточки и обеспечить всех необходимым минимумом продуктов и товаров по низким ценам; 2) создать свободный рынок (коммерческие цены), который должен был постепенно расширяться.

\hfill

Съезд народных депутатов СССР, несмотря на серьезную критику (основной и вполне, как представляется, справедливый мотив которой был следующий: за пять лет не сделали ни одного шага по радикальной перестройке экономики, переход к рыночной экономике откладывался, по существу, на шесть лет), утвердил правительственную программу. Однако уже через несколько месяцев всем стало ясно, что она не действует.

\hfill

На третьей сессии Верховного Совета СССР правительство выступило с новой программой осуществления экономической реформы (докладчик Н. И. Рыжков). В ней намечалось с 1 июля 1990 года, резкое повышение розничных цен на хлеб и хлебобулочные изделия (в 3 раза), с полной компенсацией населению потерь, значительное повышение цен на другие продукты и промышленные товары с частичной компенсацией или без нее.

\hfill

Эти предложения правительства вызвали настоящую панику в стране. В несколько дней с полок магазинов было сметено все. Как сказал о той программе правительства народный депутат П. Бунич, - это был шок без терапии.

В итоге обсуждения правительственная программа по концептуальным соображениям была отклонена и возвращена на доработку. В период между 3-й и 4-й сессиями Верховного Совета СССР шла активная проработка проблем перехода к рынку. 31 июля 1990 года состоялась встреча Президента СССР М. С. Горбачева и Председателя Верховного Совета РСФСР Б. Н. Ельцина, на которой была достигнута договоренность о разработке альтернативной программы. Была создана комиссия под руководством академика С. С. Шаталина и заместителя Председателя Совета Министров РСФСР Г. А. Явлинского.

\hfill

Таким образом, разработкой программы перехода к рыночной экономике одновременно занимались две комиссии: правительственная и Шаталина - Явлинского. Кроме того, для анализа и оценки альтернативных программ и других предложений по этим вопросам создали комиссию под руководством академика А. Г. Аганбегяна.

\hfill

Комиссия Шаталина - Явлинского выполнила поручение, подготовив, как общесоюзную, "Программу 500 дней". В качестве первого решающего шага "Программа 500 дней" предусматривала стабилизацию финансово-денежной системы и определяла конкретные меры для решения этой задачи. При этом цены на основные продукты и товары планировалось сохранить на неизменном уровне и лишь по мере стабилизации рубля они должны были "опускаться" по группам товаров, при сохранении контроля над ценами по другим товарам. Ее авторы четко расписали по периодам, - какие меры должны быть осуществлены в течение каждого из них. Это давало возможность общественности осуществлять постоянный контроль над ходом реализации "Программы 500 дней". В ней также обстоятельно и конкретно были проработаны такие принципиальные вопросы, как разгосударствление и приватизация экономики, вопросы структурной перестройки хозяйства, внешнеэкономической деятельности и валютной политики, программы социальной защиты населения.

\hfill

В начале сентября сессия Верховного Совета РСФСР в целом одобрила "Программу 500 дней", определила начало ее осуществления - 1 октября 1990 года. Затем эта программа была доложена на сессии Верховного Совета СССР.

\hfill

Одновременно был заслушан доклад Председателя Совета Министров СССР Н. И. Рыжкова о правительственной программе (хотя содержание самой программы правительства так и не было доведено до сведения широкой общественности).

\hfill

К тому времени Горбачев, долго не решавшийся на окончательный выбор стратегии и союзников, оказался под огнем жесткой критики как слева, так и справа. «Демократы» осуждали Горбачева за нерешительность и непоследовательность преобразований, а коммунистические консерваторы - за «предательство дела социализма» и «буржуазное перерождение».

\hfill

В октябре 1990 г. под давлением консерваторов и недоверия к рынку и демократам Горбачев отказался от ее поддержки. Была принята «компромиссная» программа. Складывавшаяся было, коалиция с демократами была ликвидирована.

\hfill

По предложению президента решили выработать единый, компромиссный вариант, хотя многие видные экономисты однозначно подчеркивали, что эти программы концептуально несовместимы. Ведь правительственная программа по существу на неопределенное время сохраняла административно-командные методы руководства экономикой, а переход к рыночным отношениям снова затягивался на неопределенный срок, тогда как программа Шаталина-Явлинского предусматривала создание в кратчайшие сроки всех необходимых структур и предпосылок для перехода к рынку.

\hfill

19 октября 1990 года, после предварительного обсуждения в комитетах и комиссиях, непродолжительных и в целом спокойных дебатов на сессии Верховный Совет СССР утвердил новый вариант президентской программы "Основные направления по стабилизации народного хозяйства и переходу к рыночной экономике". От имени межрегиональной группы было заявлено, что при некоторых условиях она могла бы поддержать "Основные направления...". Одним из таких условий являлось принятие мер к реорганизации и переформированию всех союзных органов управления на принципах национального единства, на межреспубликанской основе. Она в основном сохраняла логику и была близка по структуре к программе Шаталина-Явлинского, однако имела по многим позициям расплывчатый характер. Это дало основание многим специалистам оценить ее как документ скорее политического, чем экономического характера. Общий, неконкретный характер "Основных направлений", не привязывающих осуществление тех или иных крупных мер к определенным срокам, объективно сохранял возможность, в зависимости от тех или иных политических факторов, затягивания решения этих вопросов. В этом отношении представляют интерес высказанные на пресс-конференции мнения ведущих экономистов, участвовавших и руководивших разработкой программ. И. Абалкин, отвечая на вопрос, сколько потребуется времени - 500 или 5000 дней на стабилизацию экономики, высказался таким образом, что пока речь идет лишь о создании предпосылок перехода к рынку и его инфраструктуре. На создание же эффективной экономики потребуется не менее десятилетия, а возможно - жизнь целого поколения. Академик А. Г. Аганбегян уточнил, что период революционной ломки займет 1,5-2 года. Формирование потребительского рынка желательно завершить к концу 1991 года, в какой-то мере насытив его товарами, примерно в те же сроки - рынка денег и инвестицией. Несколько дольше будет создаваться рынок производства средств и еще дольше - рынок рабочей силы (ему должны предшествовать рынок жилья, снятие барьера прописки и пр.).

\hfill

В чем отличие "Основных направлений..." от программы Шаталина-Явлинского? Главное в том, что они предусматривали решение целого ряда ключевых проблем экономической реформы принципиально иными способами. Прежде всего, это касалось вопросов управления экономикой: наиболее серьезные рычаги управления оставались в руках центральной власти. В ее компетенцию входили ценовая и кредитная политика, эмиссионная деятельность, материально-техническое обеспечение государственных программ, налоговая и таможенная политика, экспорт основных видов сырья. В отличие от программы Шаталина-Явлинского, в них предусматривались более медленные темпы приватизации, не ставился вопрос о возможности перехода земли в частную собственность и объявления земель колхозов и совхозов суммой наделов их работников, сохранялись дотации убыточным предприятиям, колхозам и совхозам. Как видим, в "Основные направления..." не вошли наиболее интересные и радикальные предложения шаталинской группы. Кроме того, принятый документ включил в себя положения правительственного проекта по централизованному повышению оптовых и закупочных цен и установлению директивных процентных ставок коммерческим банком, которые ранее были крайне негативно оценены экспертами. Декларативный характер программы не дает представления о том, какие конкретные шаги предпримут президент, Верховный Совет СССР и союзное правительство для ее реализации.

\hfill

В рамках "Основных направлений..." каждая республика, а также Москва и Ленинград могли осуществлять свои варианты перехода к рыночным отношениям. Получалось так - ресурсы, финансы, валюта и др., т. е. реальная власть, оставались у Центра, а ответственность за осуществление программ должны были нести республики. Предполагалось, что президентская программа станет экономической основой разработки и подписания нового союзного договора.

\pagebreak
\subsection{Кризисное состояние советской экономики к 1991 году и пути его преодоления}

К 1989 году денежное обращение в СССР необратимо расстроилось, как и государственные финансы в целом. В стране стремительно нарастал дефицит товаров, что увеличивало инфляцию, в результате у населения к 1991 году скопилось много наличных денег, никак не обеспеченных товарным производством в СССР. Бездарная экономическая политика последнего советского десятилетия привела к распаду страны.

\hfill

Справедливости ради надо сказать, что предпосылки этого кризиса закладывались гораздо раньше, а именно денежной реформой 1961 года, которая породила понятие «дефицит», практически забытый в 50-е годы. После снятие Н.С. Хрущева проблемы в экономике удалось только «законсервировать» на какое-то время, но не устранить причины их порождавшие. Конечно неправильно полагать, что все проблемы СССР были только в экономической сфере. Нет, кризисные явления в экономике СССР были порождены прежде всего проблемами «застоя» в сфере государственного управления, которое давно нуждалось в модернизации. Основные принципы административно-командной системы в СССР были выстроены еще сразу после войны И.В. Сталиным и к середине 60-х годов уже нуждались в реформах, поскольку устарели к тому моменту и не отвечали вызовам времени. Этого не произошло по разным причинам.

\hfill

Основные причины экономического кризиса в СССР в середине и конце 80-х годов прошлого века и последующего распада страны, можно условно разделить на политические, экономические и социальные. Политические и социальные причины нас в данном труде интересуют менее чем экономические, поскольку это большая тема отдельного исследования. Мы подробнее остановимся на экономическим причинах, послуживших одной из причин развала СССР:

\begin{enumerate}
    \item стремительное падение цен на нефть на мировом рынке в 1986 году и как следствие падение доходной части бюджета СССР. К этому времени СССР уже был одним из основных поставщиком нефти и газа на мировой рынок. Уже в 70-е годы прошлого века сформировалась зависимость Госбюджета СССР от доходов нефтегазовой отрасли, образно говоря страна подсела на «нефтяную иглу» еще в то время;
    \item раздутый военно-промышленный комплекс. Совершенно не оправданное количество денежных средств, выделяемое на разработку и приобретение вооружений, раздутая по штатам армия (более 5 миллионов только в Вооруженных силах, не считая войска МВД и других силовых ведомств). Продолжающаяся война в Афганистане, которая требовала колоссальных затрат для содержания 120 тысячной группировки войск в Афганистане;
    \item общее снижение эффективности экономики, когда ряд промышленных предприятий выпускал продукцию, не пользующуюся спросом, другими словами было такое выражение «работал на склад», то есть товары лежали на складах и не пользовались спросом;
    \item стремительно растущий дефицит Госбюджета СССР, то есть превышение расходов над доходами, вызванный в том числе и падением мировых цен на газ и нефть;
    \item эмиссия значительного количества денег Госбанком СССР, в том числе в виде кредитов, предоставляемых через отраслевые банки различным предприятиям. На практике зачастую их не возвращали, что увеличивало денежную массу в экономике и обесценивало рубль. По данным Госбанка СССР обеспеченность рубля товарами к 1988 году составляла всего 30 копеек. Такое положение дел усугубляло дефицит товаров в магазинах;
    \item падение реальной производительности труда в колхозах и совхозах, поскольку их доходы мало зависели от количества проданной продукции. Зачатую часть урожая просто оставалась гнить на полях, а часть собранного урожая не была обработана и в конечном итоге тоже пропадала. Хотя в отчетности (на бумаге) все было в порядке и даже плановые показатели были перевыполнены. То же касалось и ряда промышленных предприятий, в том числе легкой и пищевой промышленности;
    \item фактически содержание ряда республик СССР, в первую очередь Закавказья, Средней Азии и Прибалтики. Огромные средства Госбюджета СССР, которые вкладывались в социально-экономическое развитие вышеуказанных республик не имели под собой никакого разумного экономического обоснования и кроме убытков ничего не приносили экономике СССР, а также значительного падения жизненного уровня в РСФСР. Сложилась парадоксальная ситуация, когда основной вклад в развитие экономики и ВВП СССР делал РСФСР и при этом в РСФСР был самый низкий уровень жизни и потребления среди всех пятнадцати союзных республик;
    \item низкий уровень производительности труда во всех отраслях промышленности по причине низкой автоматизации рабочих мест. СССР в 70-е годы стал значительно отставать от развитых стран во внедрении в промышленность достижений науки и техники, в начале и середине 80-х годов это отставание только увеличилось;
    \item значительная финансовая и материальная поддержка зарубежных стран Ближнего Востока, Африки и Латинской Америки, избравших «социалистический путь развития», поставки им продовольствия, техники и вооружения, топлива и ГСМ, зачастую на безвозмездной основе;
\end{enumerate}

В результате вышеперечисленных причин в период с 1985 года по 1991 годы стремительно нарастал государственный долг СССР вследствие дефицита Госбюджета СССР, возник дефицит продовольственных и непродовольственных товаров на прилавках магазинов, были израсходованы почти все золотовалютные резервы. К 1989 году СССР стал фактически страной-банкротом и был вынужден занимать деньги у международных валютно-финансовых организаций, прежде всего МВФ. Жизненный уровень населения стремительно снижался. Недовольство населения стали использовать внешние силы, прежде всего США и страны Западной Европы, возник ряд межнациональных конфликтов в республиках СССР.

\hfill

23 января 1991 года, на излете существования СССР, была предпринята попытка стабилизировать финансовую и денежную систему страны путем изъятия наличной денежной массы из обращения. В историю эта денежная реформа вошла как «павловская», по фамилии премьер-министра СССР В.С. Павлова, занимавшего эту должность с 14 января по 21 августа 1991 года. Правительство за три дня путем замены купюр достоинством 50 и 100 рублей изъяло из обращения значительную массу наличных денег. Обмен производился только три дня с 23 по 25 января на сумму среднемесячной заработной платы. Народу озвучили обоснование реформы - значительный ввоз фальшивых денег из-за рубежа. Более бестолкового объяснения было трудно придумать. Получилось жалкое подобие денежной реформы 1947 года, только в отличие от сталинской реформы, эта реформа буквально ограбила и сделала нищими значительную часть населения страны. И еще одновременно были подняты государственные цены на продовольственные и непродовольственные товары. Чуть больше чем через полгода, в результате бездарного управления страной М.С. Горбачевым, СССР рухнул.

\pagebreak
\subsection{Программа перехода к рынку Ельцина — Гайдара и ее реализация в 1992 году}

Основная статья: \url{https://ru.wikipedia.org/wiki/%D0%A0%D0%B5%D1%84%D0%BE%D1%80%D0%BC%D1%8B_%D0%BF%D1%80%D0%B0%D0%B2%D0%B8%D1%82%D0%B5%D0%BB%D1%8C%D1%81%D1%82%D0%B2%D0%B0_%D0%95%D0%BB%D1%8C%D1%86%D0%B8%D0%BD%D0%B0_%E2%80%94_%D0%93%D0%B0%D0%B9%D0%B4%D0%B0%D1%80%D0%B0}

\hfill

Программа перехода к рынку, известная также как экономическая программа Ельцина-Гайдара, была реализована в России в 1992 году. Она была разработана в целях перехода от централизованной плановой экономики к рыночной экономике. Программа включала в себя ряд мер, направленных на создание условий для развития свободного предпринимательства и стимулирования экономического роста.

\hfill

В основе программы лежали следующие ключевые меры:

\begin{enumerate}
    \item Либерализация цен: была проведена декриминализация цен и постепенное установление свободных рыночных цен на товары и услуги. Это позволило предприятиям определять цены на свою продукцию в зависимости от рыночных условий.
    \item Приватизация: был запущен процесс приватизации государственных предприятий. Целью приватизации было передача контроля над предприятиями из рук государства в частные руки, чтобы стимулировать эффективное хозяйствование и развитие предпринимательства.
    \item Финансовая стабилизация: в рамках программы были предприняты меры по борьбе с гиперинфляцией и финансовому кризису. Была введена новая российская валюта - российский рубль, и проведены реформы в банковской системе.
    \item Реформы налогообложения и финансирования: введение налогов на прибыль и оборот предприятий, а также сокращение государственных расходов и дотаций.
    \item Разрешение на предпринимательскую деятельность: были приняты законы, обеспечивающие свободу предпринимательства, отменены прежние ограничения на создание и развитие частного бизнеса.
\end{enumerate}

Однако реализация программы столкнулась с серьезными трудностями и вызвала значительные социально-экономические потрясения. Процесс приватизации часто сопровождался коррупцией и несправедливым распределением ресурсов. Гиперинфляция и экономический спад привели к сокращению доходов населения и ухудшению жизненного уровня для многих граждан.

\hfill

В целом, программа перехода к рынку Ельцина-Гайдара имела свои положительные и отрицательные стороны. С одной стороны, она создала основу для развития рыночной экономики и предоставила больше свободы предпринимателям. С другой стороны, многие люди столкнулись с экономическими трудностями и социальными проблемами в период перехода.

\pagebreak
\subsection{Экономическое развитие России в 1993–2010 годах: тенденции, проблемы, итоги, перспективы}

Многие цели и методы экономической политики властей, проводившейся в 1990-е годы, формировались, исходя из указаний международных финансовых организаций, в первую очередь МВФ[3].

\hfill

Макроэкономическая политика, проводившаяся в 1995—1998 годах, была в целом неудачной. Её следствием стали, в частности, спад производства и значительный отток капитала из страны. В августе 1998 года было объявлено о дефолте по российским гособязательствам и об отказе поддерживать курс рубля, что означало крах экономической политики, проводившейся с 1992 года. Экономика получила тяжёлый удар, в частности, произошёл резкий спад производства и доходов населения, всплеск инфляции. Однако спад, хотя и тяжёлый, был кратковременный и вскоре сменился экономическим ростом. В числе факторов перехода к росту были изменения в экономической политике властей, произошедшие после дефолта[3].

\hfill

Особенности антиинфляционной политики 1995—1998 годов:[3]

\begin{enumerate}
    \item В качестве основной антиинфляционной меры использовалось сокращение денежного предложения, в том числе за счёт массовых невыплат зарплат и пенсий (зарплаты выдавались денежными суррогатами, которые заполняли денежное обращение) ;
    \item Применение завышенного курса рубля с целью сокращения инфляции.
\end{enumerate}

Хотя темпы инфляции снизились, это не привело к росту инвестиций и запуску процесса модернизации экономики. Государство, применяя сомнительные методы противодействия инфляции и превратившись в крупнейшего нарушителя финансовых обязательств, внесло большой вклад в поддержку высокого уровня недоверия в экономике, что сильно препятствовало инвестиционной активности. Следствием применения завышенного курса рубля стало снижение конкурентоспособности отечественных производителей. Следствием чрезмерного снижения денежного предложения — бартеризация экономики, массовые неплатежи и т. п. явления.

\hfill

Особенности бюджетной политики 1995—1998 годов:[3]

\begin{enumerate}
    \item Финансирование дефицита госбюджета за счёт наращивания государственного долга. Причём объёмы привлечения денежных средств постоянно увеличивались. Так, объём размещения ГКО-ОФЗ вырос со 160 млрд рублей в 1995 году до 502 млрд рублей в 1997 году. Нужный объём спроса на государственные ценные бумаги поддерживался за счёт сохранения относительно высоких ставок процентов, а также за счёт привлечения спекулятивного иностранного капитала. Ориентация на последний потребовала снятия большей части ограничений на вывоз капитала.
    \item Сохранение высоких налоговых ставок для поддержания доходов госбюджета.
\end{enumerate}

Финансирование бюджетного дефицита за счёт заимствований на финансовых рынках имело ряд негативных последствий для экономики. В частности, высокая прибыльность операций с государственными ценными бумагами оттягивала финансовые ресурсы из реального сектора экономики в финансовый сектор. Ориентация расходов госбюджета на рефинансирование государственных ценных бумаг значительно сужала возможности государства по поддержанию социальной сферы и экономики страны. Кроме того, резкий рост госдолга приводил к значительному увеличению рисков, связанных с колебаниями курсов ценных бумаг и курса российского рубля. А либерализация международных операций с валютой ослабляла защиту экономики страны от внешнего давления на российский рубль и от утечки капиталов[3].

\hfill

2000-е годы в экономике России: в 2000—2008 годах в экономике России отмечался рост ВВП (в 2000 — 10\%, в 2001 — 5,7\%, в 2002 — 4,9\%, в 2003 — 7,3\%, в 2004 — 7,2\%, в 2005 — 6,4\%, в 2006 — 7,7\%, в 2007 — 8,1\%, в 2008 — 5,6\%), промышленного и сельскохозяйственного производства, строительства, реальных доходов населения[1]. Происходило снижение численности населения, живущего ниже уровня бедности (с 29\% в 2000 году до 13\% в 2007).[2][3]. С 1999 по 2007 годы индекс производства обрабатывающих отраслей промышленности вырос на 77\%, в том числе производства машин и оборудования — на 91\%, текстильного и швейного производства — на 46\%, производства пищевых продуктов — на 64\%[4]. В 2009 году на фоне мирового экономического кризиса произошёл спад ВВП на 7,9\%.

\hfill

Рост ВВП России в 2010—2011 годах составил 8,8\%. По итогам 2011 года инвестиции в России достигли рекордного за последние 20 лет уровня в 370 млрд долларов за год. Таким образом каждый день в экономику России инвестировалось более 1 миллиарда долларов. Темпы инфляции опустились до рекордно низкого уровня со времён распада СССР: за 2011 год индекс цен вырос только на 6,6\%. С 1 января 2012 года начало работу Единое экономическое пространство России, Белоруссии и Казахстана. 22 августа того же года Россия вступила во Всемирную торговую организацию.

\hfill

В 2000-е годы в России был проведён ряд социально-экономических реформ: налоговая, земельная, пенсионная (2002), банковская (2001—2004), монетизация льгот (2005), реформы трудовых отношений, электроэнергетики и железнодорожного транспорта.

\pagebreak
\subsection{Политическое развитие страны в середине 80-х годов: истоки и потребность обновления}

Политическое развитие СССР в середине 1980-х годов было связано с рядом истоков и потребностей в обновлении. Рассмотрим их подробнее:

\begin{enumerate}
    \item Экономические проблемы: В это время СССР столкнулся с серьезными экономическими проблемами, такими как низкая производительность, устаревшая промышленность, дефицит товаров и неэффективное распределение ресурсов. Эти проблемы привели к замедлению экономического роста и ухудшению жизненного уровня населения.
    \item Политическая застойность: В СССР в середине 1980-х годов был заметен политический застой. Государственная система и политический аппарат стали неэффективными и отсталыми. Это привело к потере доверия населения к власти и недовольству существующим политическим порядком.
    \item Необходимость модернизации: Чтобы справиться с экономическими проблемами и политическим застоем, СССР нуждался в модернизации и обновлении. Была потребность в реформах, чтобы преодолеть устаревшие структуры и системы и двигаться в направлении более открытой и эффективной политической и экономической системы.
    \item Гласность и перестройка: Под руководством Михаила Горбачева, который стал генеральным секретарем КПСС в 1985 году, началась политика гласности и перестройки. Гласность предполагала большую открытость и свободу слова, а перестройка включала экономические и политические реформы для улучшения состояния страны.
    \item Национальные движения и требования: В середине 1980-х годов в СССР возникли национальные движения и требования независимости от республик, составляющих СССР. Это было вызвано неудовлетворенностью местными национальными культурами и потребностью в большей автономии.
\end{enumerate}

В результате, политическое развитие СССР в середине 1980-х годов требовало модернизации и обновления. Это привело к внедрению политики гласности и перестройки, которые стали катализаторами для последующих политических и экономических изменений и, в конечном счете, к развалу СССР в 1991 году.

В марте 1985 года, после смерти К. У. Черненко, к власти в стране пришёл М. С. Горбачёв. В 1985—1986 годах Горбачёвым и его единомышленниками в руководстве проводилась политика ускорения социально-экономического развития[90] (т. н. «Ускорение») — антиалкогольная кампания, «борьба с нетрудовыми доходами», введение госприёмки.

\hfill

После январского пленума 1987 года руководством страны были начаты более радикальные реформы: фактически, новой государственной идеологией была объявлена «перестройка» — совокупность экономических и политических преобразований, результатом которых стала резкая дестабилизация общественно-политической и экономической жизни страны, разрушение советского строя, переход к капитализму и распад СССР. В результате политики «перестройки» СССР в начале 90-х годов остался в Европе вне каких-либо политических альянсов[24].

\hfill

В ходе перестройки (со второй половины 1989 года, после первого Съезда народных депутатов СССР[91]) резко обострилось политическое противостояние сил, выступающих за социалистический путь развития, и движений, связывающих будущее страны с организацией жизни на принципах капитализма, а также противостояние по вопросам будущего облика Советского Союза, взаимоотношений союзных и республиканских органов государственной власти и управления.

\pagebreak
\subsection{Политическая реформа 1988 года: замысел и результаты}

Летом 1988 года на XIX Всесоюзной партконференции Горбачев предложил ряд мер по перестройке политической системы:

\begin{enumerate}
    \item Создать новый орган власти – Совет народных депутатов;
    \item Сделать выборы альтернативными (раньше они были безальтернативными: люди голосовали «за» или «против» одного, предложенного властью, кандидата).
\end{enumerate}

Реакция делегатов партконференции была неоднозначной:

\begin{enumerate}
    \item часть делегатов высказалась за то, что необходимо отказаться от проведения политических реформ, ограничить гласность, свернуть процессы демократизации, поскольку они ставят под угрозу завоевания социализма;
    \item другая часть высказалась за то, что необходимо действовать более решительно, провести последовательные демократические преобразования, разрешить реальную многопартийность, провести свободные альтернативные выборы, ликвидировать цензуру, признать идеологические многообразие, включая право на существование идеологий, оппозиционных коммунистической.
\end{enumerate}

Мнение вторых возобладало. В 1989 году состоялись выборы народных депутатов СССР по новому избирательному закону.
В мае – июне 1989 года прошел I съезд народных депутатов СССР. М. С. Горбачев был избран на должность Председателя Верховного Совета СССР.

\hfill

На I Съезде народных депутатов была сформировала Межрегиональная депутатская группа (МДГ), во главе с Сахаровым и Ельциным. Эта группа депутатов выступала за дальнейшую либерализацию и демократизацию политической системы в СССР.

\hfill

МДГ своего добилась на III Съезде народных депутатов в марте 1990 года. Из Конституции СССР была убрана статья 6 («Руководящей и направляющей силой советского общества, ядром его политической системы, государственных и общественных организаций является Коммунистическая партия Советского Союза»). Это событие открыло дорогу для формирования многопартийной системы.

\hfill

Помимо этого, на III Съезде Горбачев был избран депутатами Президентом СССР.

\hfill

Провозглашение политики гласности произошло в январе 1987 года.

Гласность – это ослабление цензуры и стремительный рост количества средств массовой информации разной направленности.

Последствия Гласности:

\begin{enumerate}
    \item Реабилитация репрессированных 1920 – 1950 годов.
    \item Публикация работ эмигрантов «третьей волны» (1960 – 1970-е годы): Бродского, Солженицына.
    \item Принятие Декларации о незаконности и преступности сталинской политики насильственного переселения народов.
\end{enumerate}

Итоги Гласности:

\begin{enumerate}
    \item Стремление к полной информированности народа.
    \item Признание кризиса системы.
    \item Резкому столкновению идейных, социальных, политических, национальных, религиозных течений, группировок.
\end{enumerate}

\pagebreak
\subsection{Возрождение многопартийной системы: предпосылки и особенности}

В ходе перестройки, по мере утраты политической инициативы КПСС, в стране усилился процесс формирования новых политических сил. Начала формироваться многопартийность.

В мае 1988 года была создана первая партия – «Демократический союз». Она объявила себя оппозиционной к КПСС.

В апреле 1988 года в прибалтийских республиках возникли народные фронты, которые стали независимыми массовыми организациями. Позже такие же фронты возникли в других республиках СССР.

\hfill

В течение 1998-1990 появилось множество других партий, отражавших весь спектр политической жизни страны.
Либеральное направление было представлено «Демсоюзом», христианскими демократами, конституционными демократами,  республиканцами и т.д. Наиболее крупной из либеральных партий стала Демократическая партия России, созданная в 1990 году.

Социалистические и социал-демократические идеи отстаивали Социал-демократическая партия России, Социалистическая партия.

Стали появляться и националистические политические партии. В такие партии были преобразованы народные фронты прибалтийских и некоторых других республик СССР.

Основная политическая борьба развернулась между коммунистами и либералами (демократами).

Либералы выступали за приватизацию государственной собственности, свободу личности, переход к рыночной экономике. Коммунисты были категорически против частной собственности и рыночной экономики, которые они считали основой капитализма.

\hfill

В марте 1990 года III Съезд народных депутатов отменил 6-ю статью Конституции СССР, которая законодательно утверждала руководящую роль КПСС. Были приняты другие поправки в Конституцию СССР, допускаюшие многопартийность.

В октябре 1990 года принят закон СССР «Об общественных объединениях». Стала возможной официальная регистрация политических партий. Первыми официально зарегистрированными партиями в марте 1991 года стали Демократическая партия России (ДПР), Социал-демократическая партия России (СДПР) и Республиканская партия Российской Федерации (РПРФ).

\end{document}